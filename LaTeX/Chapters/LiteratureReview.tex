\section{Literature Review}

Malware analysis is a complex and ever-changing topic, containing many features to train a machine learning model upon. As (Schultz et al., 2000) proposes, “Eight to ten malicious programs are created every day, and most cannot be accurately detected until signatures have been generated for them” indicating a huge necessity for real-time analysis to be made to prevent infection on host devices.  
This is supported by Symantec, an industry revolutioniser in commercial Anti-Virus engines, whom report of “[an increase] in the first six months of 2017, Symantec blocked just over 319,000 ransomware infections” (Symantec, 2017) evidently, Malware, especially ransomware is on the dramatic increase. Case studies such as WannaCry “with infections recorded across 150 countries globally” (Nominet, 2017) lightly demonstrate the global and non-target-specific gripe the threat poses. 

  
Thankfully, there has been some continued investigation into circumventing the limitations that signature-based detection retains. For example, (Schultz et al., 2000) created a framework whom “automatically extracted a binary profile from each example in [their] dataset” using “properties … such as byte sequences”. This is an effective alternative to other frameworks proposed by researchers such as (Tesauro, Kephart and Sorkin., 1996) who only achieved a small detection rate due to only investigating PC Boot-sector viruses, of which “PC boot sectors are 512 bytes long”. With this limited amount of code, there is a very minimal expectation of the features that can be extracted.  
When larger datasets containing a lot of features are created, such as that in (Mohaisen, Alrawi and Mohaisen, 2015) Consisting of a much larger scale of “115,157 malware samples”.  


Whilst the study “used only a total of 65 features for classification and clustering”, their best performing algorithm achieved a 85 accuracy rate. However, it should be noted that their performance scale was calculated on both the highest yet most time-efficient algorithm. Other explored frameworks may have achieved a higher accuracy score, however at a much higher systemresource cost.  
This is a noticeable improvement over alternatively suggested frameworks and algorithms such as those of (Bayer et al, 2009), whom suggest that “aggressive approximate clustering techniques may need to be employed [with much larger datasets]” resulting in a loss of accuracy due to generalization. The dramatic decrease in accuracy as cluster – hence dataset sizes increases is shown in Figure 2 


Arguably, system-resource usage is an expendable commodity in order to achieve a higher classification detection rate, especially in a corporate environment. Although, a much larger corporate environment such as an Enterprise who faces a vast quantity of advanced persistent threats from malicious actors daily. 
Binary classification appears to be a popular and successful approach to solving the issue of Malware classification. Binary classification, within the context of extracting information of Malware involves the “process of classifying given document/account on the basis of predefined classes” (Kumari and Kr., 2017). 