%\vspace{-2em}
\section{Overview}

Whilst the idea of combining Computing resources and technical knowledge to solve both human social and scientific issues isn't a new concept, the introduction of Machine Learning and the use of Artificial Intelligence to solve problems of a grandeur scale is at an unprecedented level. 


The applicability of AI can be seen as implemented at all scales of suitability. Companies such as Facebook and Twitter use Artificial Intelligence to learn more about their Users at an individual level, tailoring the content that is delivered that is relevant to their interests. Or from a business perspective, deliver customised advertising that will bring a much higher rate of ad-revenue because of the Users interests.  



Considering, this use of AI is arguably trivial when comparing other implementations of the ground-breaking technology. Ranging from self-driving Cars to progressing the very forefront of Science with Quantum Computing and bio-medical prediction. (Chui et al., 2019), 
This research project will investigate and apply various machine-learning algorithms and models against a collated data-set of Portable Executable's, henceforth referenced as PE files for identification of their behaviour and intent to a host Operating System  
Microsoft Windows’ PE file is a standard of file-formatting, insisting the structure of various objects across the family of Windows Operating System releases.  



This PE structure is a prevalent rule-set that can be found across many file-types; ranging from executable files for applications, to text-font files and object files - all essential for functionality of the Operating System.  
The Windows PE File comprises of eleven sections. Whilst application Authors can populate these sections as desired, each section will contain metadata, informing the Operating System of how the file should be processed within the application (e.g. defining it as a executable file rather than an object file) (Windows Developer Center, 2019). 

Whilst the Windows Operating System family has evolved, the PE file has continued with little variation.

\par

Microsoft's Windows Operating System has done a remarkable job at maintaining compatibility with previous generations of the Windows Operating System. The MS-DOS header within a PE file is an excellent example of these efforts of backwards compatibility. This header enables the Operating System to process the file and check for compatibility across the whole file. Where there is no compatibility, rather than just crashing as the first editions of MS-DOS used too, the Operating System will now just inform the User that there is no current compatibility. 

\newpage

\section{Problem Statement}


This research project will highlight the limitations of current mitigations in place to combat the threat of Malware. Signature-based identification is the first-line of defence against the war on Malware. 
Signature-based protection solely relies upon the malignant intent of a sample file already being identified. This could be through various channels such as manual inspection from an Analyst investigating the files behaviour to the pre-cursor knowledge of the behaviours the file exhibits, i.e. the file has performed malicious actions on a previous device. 
Due to the biological characteristics of malicious code, the identification of brand-new or “zero-day” variants are an on-going "Cat and Mouse" game between both Malware Analyst and Malicious threat actor. These threat actors can create a substrate of a pre-existing Malware variant in very little time - in comparison to the resources needed for a Malware Analyst to determine the heuristics and behaviours of such generated code. 
As Signature-based detection effectively compares a file’s “fingerprint” or signature against a predetermined list. These signatures are created from a bit-to-bit accuracy, where any deviation of for example, a character will attribute the file with a new signature. This is extremely problematic in combatting Malware, as although the malicious intent of the file remains the same, due to fact there is a slight deviation, an anti-virus engine using Signature-based detection will treat it as a safe file. 


In the context of Artificial Intelligence, classification is a quintessential concept. From detecting animals to the likes of human expression. The proposed research project explores pre-existing classification concepts to create an Artificial Intelligence model capable of detecting new substrates of malicious code “on the fly” as opposed to traditional, current capabilities.

\section{Research Questions}


\begin{itemize}
	\item What attributes can be found within malicious code that can be used to identify its intent? 
	\item Do variants of malicious code have similar attributes, and can these common characteristics be applied to the same machine-learning model used for classification?
	\item Can these attribute identification be accurate enough for a true-positive classification, in comparison to traditional methods of classification such as signature-based detection? 
	\item What signatures and characteristics do current Anti-Virus engines use to identify and classify variants of malicious code?
	\item Can any machine-learning model be used to combat never-seen before substrates of malicious code from knowledge of pre-existing Malware?
\end{itemize}

\newpage

\section{Research Objectives}


To achieve the aim laid out for this research project, the following objectives will be investigated:

\begin{table}[!ht]
	\centering
	% To place a caption above a table
	\caption{Defining the Research Objectives of this Paper.}
	\label{table:1}
	\begin{tabular}{lllll}
		\cline{1-2}
		\multicolumn{1}{|l|}{1} & \multicolumn{1}{l|}{\begin{tabular}[c]{@{}l@{}}To investigate the characteristics of malicious code and\\   how these attributes can be used by a Machine Learning model for\\   identification.\end{tabular}}                                                                 &  &  &  \\ \cline{1-2}
		\multicolumn{1}{|l|}{2} & \multicolumn{1}{l|}{\begin{tabular}[c]{@{}l@{}}To assess pre-existing Machine Learning Models and Malware\\   identification techniques and understand their effectiveness in identifying\\   malicious code.\end{tabular}}                                                    &  &  &  \\ \cline{1-2}
		\multicolumn{1}{|l|}{3} & \multicolumn{1}{l|}{\begin{tabular}[c]{@{}l@{}}To evaluate the reliability and accuracy of a variety of\\   developed Machine Learning Models and their performance in detecting\\   malicious heuristics of code, identifying and preventing further infection.\end{tabular}} &  &  &  \\ \cline{1-2}
		&                                                                                                                                                                                                             &  &  & 
	\end{tabular}
\end{table}
