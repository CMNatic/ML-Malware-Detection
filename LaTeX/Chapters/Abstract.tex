The prevalent risk of malicious infection is an ever-increasing threat to the availability of World-wide Information Technology infrastructure, whom form as pillars to the very day-to-day functioning of society as we know it.

Presenting in many forms, malware variants have a variety of common objectives from an attacker’s perspective. One category known as "Ransomware", is a variant of malicious code whom prohibits a user from accessing a system and/or its data until a ransom is paid - a modern twist on the traditional extortion schemes used by criminal syndicates such as highwaymen.

Modern-day Anti-Virus packages from companies such as Avast, McAfee and Microsoft's Windows Defender heavily rely on the detection of malware based upon known behaviours or features from previous infection. Known as signature-based detection, this extensively used technique is arguably a reaction-based approach - rather then proactive response to combatting this emerging threat.

The following research project investigates and experiments with applying machine-learning models to be able to score the maliciousness of code within datasets based upon their heuristics, where the models have no prior-knowledge of the file presented.